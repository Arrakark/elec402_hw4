\documentclass[11pt]{article}
\usepackage{amsmath, amssymb, amscd, amsthm, amsfonts}
\usepackage{graphicx}
\usepackage{hyperref}
\usepackage{subfigure}
\usepackage{float}
\usepackage{rotating}
\usepackage{tikz}
\usepackage{xcolor}
\usepackage{listings}
\definecolor{vgreen}{RGB}{104,180,104}
\definecolor{vblue}{RGB}{49,49,255}
\definecolor{vorange}{RGB}{255,143,102}
\lstdefinestyle{verilog-style}
{
    language=Verilog,
    basicstyle=\small\ttfamily,
    keywordstyle=\color{vblue},
    identifierstyle=\color{black},
    commentstyle=\color{vgreen},
    numbers=left,
    numberstyle=\tiny\color{black},
    numbersep=10pt,
    tabsize=4,
    moredelim=*[s][\colorIndex]{[}{]},
    literate=*{:}{:}1
}

\makeatletter
\newcommand*\@lbracket{[}
\newcommand*\@rbracket{]}
\newcommand*\@colon{:}
\newcommand*\colorIndex{%
    \edef\@temp{\the\lst@token}%
    \ifx\@temp\@lbracket \color{black}%
    \else\ifx\@temp\@rbracket \color{black}%
    \else\ifx\@temp\@colon \color{black}%
    \else \color{vorange}%
    \fi\fi\fi
}
\makeatother

\usepackage{trace}



\oddsidemargin 0pt
\evensidemargin 0pt
\marginparwidth 40pt
\marginparsep 10pt
\topmargin -20pt
\headsep 10pt
\textheight 8.7in
\textwidth 6.65in
\linespread{1.2}

\title{NAND3 Simulation and Layout - Delay Analysis}
\author{Vladislav Pomogaev - 26951160}
\date{November 14, 2021}

\newcommand{\rr}{\mathbb{R}}

\newcommand{\al}{\alpha}
\DeclareMathOperator{\conv}{conv}
\DeclareMathOperator{\aff}{aff}

\begin{document}

\maketitle

\section{Problem 1}
\subsection{Specifications}
Here are the specifications for my NAND3 gate:
\begin{itemize}
\item Area: 0.0981 micrometer squared
\item Average delay: 25.61 pico-seconds
\item Area delay: 2.5130 pico-seconds micrometer squared
\end{itemize}

\begin{figure}[H]
	\includegraphics[width=0.98\textwidth]{layout_annot.png}
  	\caption{Layout of the gate. I had to add bulk connections otherwise I could not pass LVS for some reason. For the area, I measured the largest dimensions made by the active areas.}
\end{figure}

\begin{figure}[H]
  \includegraphics[width=0.98\textwidth]{drc.png}
  \caption{DRC Summary report (I solved the two rule checks that we were allowed to ignore)}
\end{figure}

\begin{figure}[H]
  \includegraphics[width=0.98\textwidth]{timing.png}
  \caption{Timing of the gate. td,HL = 25.507ps, td,LD = 25.72ps. This graph was generated from the extracted view including all parasitics.}
\end{figure}

\begin{figure}[H]
  \includegraphics[width=0.98\textwidth]{schem.png}
  \caption{Screen-shot of my test bench schematic. I used a 10pF capacitor as the load. Worst-case switching was used (switching C, aka bottom n-fet results in worst switching).}
\end{figure}

\section{Problem 2}
\subsection{a)}
Using demorgan's law:
\[
\overline{F} = C D  (B + A)
\]
\[
\overline{\overline{F}} = \overline{C  D  (B + A)}
\]
\[
F = \overline{C} + \overline{D}  + \overline{B + A}
\]
\[
F = \overline{C} + \overline{D}  + \overline{B} \overline{A}
\]

\subsection{b)}

\begin{figure}[H]
\centering
  \includegraphics[width=0.8\textwidth]{q2size.png}
  \caption{Drawing indicating the sizing of the gates when the output resistance is the same of that of a 8:4 P-N inverter.}
\end{figure}

\subsection{c)}

\begin{figure}[H]
\centering
  \includegraphics[width=0.8\textwidth]{q2worstcase.png}
  \caption{Two tables indicating the worse-case switching conditions for this gate. T=0 is the initial condition, T=1 is the second condition. These switching conditions switch the largest charged capacitances through the largest resistances.}
\end{figure}

In order to verify that these were in fact the worst-case conditions for this gate, I decided to compare them with what I thought were the second-worst case switching conditions for both $t_{PHL}$ and $t_{PLH}$. Here are what I think the second worst-case scenarios are. 


\begin{figure}[H]
\centering
  \includegraphics[width=0.8\textwidth]{q22worstcase.png}
  \caption{Two tables indicating the \textbf{second} worse-case switching conditions for this gate. These conditions switch all but one set of capacitances. }
\end{figure}

Now I made this circuit in Cadence, and simulated all of the various switching conditions.


\begin{figure}[H]
\centering
  \includegraphics[width=0.8\textwidth]{schem2.png}
  \caption{Schematic of the gate.}
\end{figure}

\begin{figure}[H]
\centering
  \includegraphics[width=0.8\textwidth]{schemtb2.png}
  \caption{Test bench schematic. The signals generated for A, B, C, and D were generated depending on the test being run. They had a rise time of 0.}
\end{figure}

\begin{figure}[H]
\centering
  \includegraphics[width=0.8\textwidth]{elmore.png}
  \caption{Elmore delay for this circuit, for figuring out worst-case switch conditions.}
\end{figure}


\begin{figure}[H]
\centering
  \includegraphics[width=0.8\textwidth]{wctphl.png}
  \caption{Switching waveform for worst-case $t_{PHL}$. The switching delay is 2.227ps}
\end{figure}


\begin{figure}[H]
\centering
  \includegraphics[width=0.9\textwidth]{wctplh.png}
  \caption{Switching waveform for worst-case $t_{PLH}$. The switching delay is 1.889ps}
\end{figure}

\begin{figure}[H]
\centering
  \includegraphics[width=0.9\textwidth]{swctphl.png}
  \caption{Switching waveform for \textbf{second} worst-case $t_{PHL}$. The switching delay is 2.108ps}
\end{figure}

\begin{figure}[H]
\centering
  \includegraphics[width=0.9\textwidth]{swctplh.png}
  \caption{Switching waveform for \textbf{second} worst-case $t_{PLH}$. The switching delay is 1.644ps }
\end{figure}

Tallying it all up, here are the switching delays:
\begin{center}
\begin{tabular}{ c c c }
   & $t_{PHL}$ & $t_{PLH}$ \\ 
 Worst-case & 2.227ps & 1.889ps \\  
 Second-worst-case & 2.108ps & 1.644ps    
\end{tabular}
\end{center}

As we can see, the worst-case delay is longer than the second-worst case delay. While this isn't exhaustive proof that we have in fact found the worst-case delay, it serves to reinforce my understanding of where the delays are coming from.

\section{Problem 3}
\subsection{a)}

\begin{figure}[H]
\centering
  \includegraphics[width=0.9\textwidth]{q3.png}
  \caption{Derivation of the output expression. Here I made the assumption that sel and selB are inverses of each other.}
\end{figure}


\end{document}